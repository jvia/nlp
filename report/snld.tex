\documentclass[12pt]{article}
\usepackage{times}
\usepackage{latexsym}
\usepackage{xspace}
\usepackage{times}
\usepackage{epsfig}
\setlength{\topmargin}{0 mm}
\setlength{\headsep}{0 mm}
\setlength{\headheight}{0 in}
\setlength{\voffset}{0 mm}
\setlength{\oddsidemargin}{0 mm}
\setlength{\evensidemargin}{0 mm}
\setlength{\hoffset}{0 mm}
\setlength{\textwidth}{6.5 in}
\setlength{\textheight}{9 in}

\title{Introspection Guided Dialogue-Based Task Resolution}
\author{Jeremiah Via}

\begin{document}
\maketitle

\begin{abstract}
  Situated task-based dialogue poses many challenges for traditional
  natural language understanding techniques. Key among these are
  ambiguous reference resolution, incomplete utterances, task-specific
  utterances. This paper presents a probabilistic belief framework for
  dealing with these challenges. By representing beliefs about the
  world, the agent's own capabilities, and the capabilities of other
  agents in the environment, a dialogue manager can choose actions
  which have the highest likelihood of increasing common ground. A
  probabilistic approach has the additional benefit of remaining
  robust to contradictions, allowing the dialogue manager to simply
  update its probabilities.
\end{abstract}

\section{Introduction}
From Matthias' notes:
\begin{itemize}
\item Ambiguities
  \begin{itemize}
  \item Syntactic ambiguity: “right next to the doorway on the left on the floor” (and ambiguous confirmation “towards the second room”)
  \item Prosody-based confirmation: “yeah that doorway” (without knowing whether the director got it right)
  \item Repeated confirmation: “okay”, “um okay” (nothing new)
  \end{itemize}
\item Multi-phrase expressions
  \begin{itemize}
  \item Sentence instead of referent: “the – so I went ...” (also, not
    clear when sentence ends, i.e., where the “on the bookshelf in the
    corner of the room” belongs)
  \item Cognitive pauses: “uh” and “um” (to gain time)
  \item Assumption about map: “on the bookshelf in the corner” (there
    is only one, corner does not have to be specified)
  \end{itemize}
\item Omissions
  \begin{itemize}
  \item Omissions and insertions: “uh I guess” inserted, “on” and
    “you” omitted
  \item Task-based reference: “put one in” and “that's all of them”
    (assumes “yellow block”)
  \item Contraction: “all of them you have” (not about “having” them,
    but about “having put them in pink boxes” )
  \end{itemize}
\item Task knowledge
  \begin{itemize}
  \item Superfluous words: “like” without meaning, omit for parsing and interpretation
  \item Perspective-taking: “the first room” or “keep going straight” (assumes knowledge of the other's orientation or position in the world)
  \item Task goals: “unless you see anything” (assumes knowledge of the task; clearly, “anything” does not mean “anything”, it means “green box” or “blue box” based on the task instructions)
  \end{itemize}
\end{itemize}

%  ungrammatical sentences (incomplete referential phrases, missing verbs, corrections, ...)
%  wrong word substitutions for intended target words (“block” and “book” for “box”)
%  underspecified directions, referents, and directives (assumes shared task-knowledge, knowledge of subgoals, perspectives, etc.)
%  lots of “ums and “uhs” indicating cognitive load
%  lots of coordinating “okays” (at least three different kinds can be distinguished by prosody: for acknowledgment of understanding, for requesting expansion, and for acknowledgment of completed action)
%  automatic gesturing and pointing by the member (even though the director being able to see the member)
%  information about meaning encoded in timing of utterances


% Introduce the reader not familiar with your project to the problem in
% situated natural language dialogues with robots that your are tackling
% in the paper in a very general way and line out what you are
% tackling/solving in your project Do not go into details, but stay at a
% very high conceptual level, address also briefly why your proposed
% algorithms are important, what they show/imply, and what the
% advantages of your approach are (as opposed to other existing
% approaches) End your introduction with a very short outline of the
% rest of the paper (in one paragraph)

In situated contexts, natural language takes on a different set of
characteristics than it does in the form in which it is typically
studied---namely prose. This makes much research not fully-applicable
to situated contexts, where natural language understanding software
must deal with disfluencies, common-ground alignment, and
response-time requirements.

Key points: algorithm uses probabilistic belief, making it more robust
to the problems posed by situatedness. depending on its confidence
about beliefs, it asks questions it believes will yield the most
information to the interlocutor it believes can yield the most
information.

\section{Scenario/Task description}
% Describe the scenario/task that you are considering in detail, make
% reference to all the sources you used and inform the reader about
% possible strategies for developing algorithms for the task

% Include background information about the task and/or other approaches
% here Describe the task and possible solutions, and point out their
% advantages and disadvantages

% You should not repeat details of well-known algorithms; instead just
% refer to them where appropriate (e.g., to the literature used in
% class) You may assume that the reader has some knowledge about
% situated natural language processing
The task is one which the robot must search for and find a red medical
kit in an office-like environment. The robot has beliefs about the
usual location of these kinds of medical kits, but when reality fails
to meet expectations, it must gather information from two
interlocutors participating in the task. Neither interlocutor has full
knowledge of the task and so the robot must direct the conversation
such that it can maximize its knowledge related to the task.

\section{Your solution}
\subsection{Description of your solution}
% Describe in detail your solution to the problem, possibly including
% graphs and figures to aid the reader Include detailed descriptions of
% your the algorithm and how the situated context is utilized to make
% parts of the NL chain work better (this is critical and should be the
% highlight of the paper)

\subsection{Experiments and results}
% Show experimental data from runs with the integrated system in ADE
% that demonstrate your improvements and how they work

% Compare your results to results without the improvements (this is
% critical because you need to show that without utilizing the
% situatedness of the agent, established algorithms would do worse --
% make sure that you have a clear performance measure defined) Perform
% simple statistics on the results if applicable S

\subsection{Analysis--how and why the solution works}
% Discuss the advantages and possible disadvantages of your algorithm
% Suggest additional experiments and/or improvements to the solution.

\section{Conclusion}
% Summarize what your solution achieved, and how your algoritm could be
% extended or used in other tasks Briefly discuss directions for future
% work.

\bibliographystyle{apalike}
{\small \bibliography{references}}
\end{document}
% LocalWords:  disfluencies
